
\begin{table}[htp]
\renewcommand{\arraystretch}{1.15}
\setlength{\tabcolsep}{5pt}
\caption{Efast sensitivity results}
\label{tab:efast}
\footnotesize
\centering
\begin{threeparttable}
\begin{tabular}{lccc}
\hline
\addlinespace
& Proportion Obese & Average BMI & SD BMI \\
\addlinespace
\hline
\addlinespace
\multicolumn{4}{l}{\textbf{Scenario 1 (only genes, N=6000)}} \\
\addlinespace[6pt]
\multicolumn{4}{l}{\hspace{1em} S1} \\
\hspace{1.5em} random-mating & 0.109 [0.105; 0.113]   & 0.097 [0.093; 0.102]   & 0.148 [0.144; 0.153] \\
	  \hspace{1.5em} genetic-variance & 0.481 [0.477; 0.485]   & 0.506 [0.502; 0.511]   & 0.308 [0.304; 0.312] \\
	  \hspace{1.5em} fertility-differential & 0.189 [0.185; 0.193]   & 0.227 [0.224; 0.230]   & 0.078 [0.074; 0.081] \\
	 \\
\addlinespace[12pt]
\multicolumn{4}{l}{\hspace{1em} ST} \\ 
\hspace{1.5em} random-mating & 0.230 [0.212; 0.249]   & 0.185 [0.167; 0.202]   & 0.521 [0.502; 0.540] \\
	  \hspace{1.5em} genetic-variance & 0.685 [0.666; 0.703]   & 0.664 [0.645; 0.682]   & 0.721 [0.702; 0.739] \\
	  \hspace{1.5em} fertility-differential & 0.337 [0.318; 0.356]   & 0.350 [0.334; 0.367]   & 0.397 [0.379; 0.416] \\
	 \\ 
\addlinespace[12pt]
    \multicolumn{4}{l}{\textbf{Scenario 2 (only vertical transmission, N=6000)}} \\
    \addlinespace[6pt]
    \multicolumn{4}{l}{\hspace{1em} S1} \\
\hspace{1.5em} random-mating & 0.022 [0.019; 0.025]   & 0.033 [0.030; 0.037]   & 0.003 [-0.000; 0.006] \\
	  \hspace{1.5em} leakage & 0.464 [0.460; 0.468]   & 0.533 [0.530; 0.537]   & 0.396 [0.392; 0.400] \\
	  \hspace{1.5em} fertility-differential & 0.048 [0.044; 0.052]   & 0.106 [0.102; 0.110]   & 0.097 [0.093; 0.101] \\
	 \\
\addlinespace[12pt]
\multicolumn{4}{l}{\hspace{1em} ST} \\ 
\hspace{1.5em} random-mating & 0.342 [0.324; 0.360]   & 0.272 [0.253; 0.291]   & 0.358 [0.336; 0.379] \\
	  \hspace{1.5em} leakage & 0.887 [0.865; 0.909]   & 0.831 [0.812; 0.850]   & 0.842 [0.823; 0.861] \\
	  \hspace{1.5em} fertility-differential & 0.441 [0.421; 0.461]   & 0.379 [0.358; 0.399]   & 0.520 [0.500; 0.539] \\
	 \\ 
\addlinespace[12pt]
    \multicolumn{4}{l}{\textbf{Scenario 3 (genes and vertical transmission, N=8000)}} \\
    \addlinespace[6pt]
    \multicolumn{4}{l}{\hspace{1em} S1} \\
\hspace{1.5em} random-mating & 0.117 [0.112; 0.122]   & 0.047 [0.042; 0.053]   & 0.144 [0.139; 0.149] \\
	  \hspace{1.5em} genetic-variance & 0.195 [0.190; 0.200]   & 0.100 [0.095; 0.105]   & 0.207 [0.201; 0.212] \\
	  \hspace{1.5em} vertical-trans-variance & 0.207 [0.202; 0.212]   & 0.092 [0.086; 0.097]   & 0.212 [0.207; 0.216] \\
	  \hspace{1.5em} fertility-differential & 0.119 [0.114; 0.124]   & 0.046 [0.040; 0.052]   & 0.114 [0.109; 0.119] \\
	 \\
\addlinespace[12pt]
\multicolumn{4}{l}{\hspace{1em} ST} \\ 
\hspace{1.5em} random-mating & 0.366 [0.343; 0.389]   & 0.638 [0.619; 0.657]   & 0.422 [0.401; 0.444] \\
	  \hspace{1.5em} genetic-variance & 0.501 [0.478; 0.523]   & 0.721 [0.700; 0.741]   & 0.505 [0.485; 0.524] \\
	  \hspace{1.5em} vertical-trans-variance & 0.621 [0.602; 0.640]   & 0.751 [0.727; 0.775]   & 0.577 [0.553; 0.600] \\
	  \hspace{1.5em} fertility-differential & 0.291 [0.268; 0.313]   & 0.577 [0.552; 0.601]   & 0.240 [0.220; 0.260] \\
	 \\
\addlinespace
\hline
\end{tabular}
\begin{tablenotes}
\scriptsize
\item 95\% confidence interval in brackets.
\item S1 = First-order indices,  measures the contribution to the output variance by a single model input alone.
\item ST = Total-order index, measures the contribution to the output variance caused by a model input, including both its first-order effects (the input varying alone) and all higher-order interactions.
\end{tablenotes}
\end{threeparttable}
\end{table}