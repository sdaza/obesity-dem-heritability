
\begin{table}[htp]
\renewcommand{\arraystretch}{1.3}
\setlength{\tabcolsep}{5pt}
\caption{Sobol sensitivity results only genes, Scenario 1 (N=16384)}
\label{tab:sobol-snp}
\footnotesize
\centering
\begin{threeparttable}
\begin{tabular}{lccc}
\hline
\addlinespace
& Proportion Obese & Average BMI & SD BMI \\
\addlinespace
\hline
\addlinespace[6pt]
\multicolumn{4}{l}{\hspace{1em} S1} \\
\hspace{1.5em} random-mating & 0.112 [0.082; 0.141]   & 0.102 [0.076; 0.128]   & 0.149 [0.088; 0.209] \\
	  \hspace{1.5em} genetic-variance & 0.494 [0.436; 0.552]   & 0.520 [0.472; 0.569]   & 0.325 [0.266; 0.384] \\
	  \hspace{1.5em} fertility-differential & 0.187 [0.152; 0.222]   & 0.225 [0.190; 0.260]   & 0.072 [0.028; 0.117] \\
	 \\
\addlinespace[12pt]
\multicolumn{4}{l}{\hspace{1em} S2} \\ 
\hspace{1.5em} random-mating x genetic-variance & 0.076 [0.026; 0.126]   & 0.050 [0.000; 0.100]   & 0.118 [0.007; 0.228] \\
	  \hspace{1.5em} random-mating x fertility-differential & 0.020 [-0.029; 0.070]   & 0.014 [-0.028; 0.055]   & 0.044 [-0.089; 0.177] \\
	  \hspace{1.5em} genetic-variance x fertility-differential & 0.088 [0.012; 0.165]   & 0.073 [0.003; 0.143]   & 0.097 [-0.030; 0.224] \\
	 \\
\addlinespace[12pt]
\multicolumn{4}{l}{\hspace{1em} ST} \\ 
\hspace{1.5em} random-mating & 0.232 [0.205; 0.259]   & 0.185 [0.168; 0.203]   & 0.487 [0.421; 0.552] \\
	  \hspace{1.5em} genetic-variance & 0.688 [0.630; 0.747]   & 0.667 [0.618; 0.717]   & 0.763 [0.666; 0.860] \\
	  \hspace{1.5em} fertility-differential & 0.330 [0.299; 0.361]   & 0.343 [0.309; 0.377]   & 0.458 [0.382; 0.534] \\
	 \\
\addlinespace
\hline
\end{tabular}
\begin{tablenotes}
\scriptsize
\item 95\% confidence interval in brackets.
\item S1 = First-order indices,  measures the contribution to the output variance by a single model input alone.
\item S2 = Second-order indices,  measures the contribution to the output variance caused by the interaction of two model inputs.
\item ST = Total-order index, measures the contribution to the output variance caused by a model input, including both its first-order effects (the input varying alone) and all higher-order interactions.
\end{tablenotes}
\end{threeparttable}
\end{table}